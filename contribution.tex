\subsection{Our Contribution}
\label{subsec:contri}


In this work, we study the feasibility of SPMD to derive 
accurate {\it component-wise power models} for modern smartphones 
which is  critical to many energy diagnostics solutions such as energy profiling~\cite{flinn:1999,shye2009into,zhang2010accurate,sesame:2011,pathak2012energy,appscope,ding2017gfxdoctor,facebookbatterymetrics}.

In order to show the in-depth analysis within  the  page limitations, 
we focus our feasibility study limited to two of the power-hungry 
phone components, the CPU and GPU, on representative phones from two generations, Moto Z3 and Nexus 6. 

Our key findings are summarized as follows:
\begin{itemize}
\item To understand the granularity of app run for SPMD application, 
we start with a measurement study of the power model dependence on
app usage. Our results show that both the CPU and the GPU of the 
two phones exhibit significant power state variations, \ie their 
power draw for the same power state (for same CPU frequency and same 
GPU frequency and state) can vary up to  13.5\% and 29.7\% for CPU and GPU, respectively, 
even when running different scenarios of the same apps. 
This result suggests that (1) any power model derived via 
TPMD offline which is based on a given microbenchmark and
can be inaccurate when it is used to model the power draw of 
different apps or different app scenarios;
(2) different scenarios of the same app should not be used together in SPMD. 
    
\cut{
\item The built-in power sensor readings available in 
modern smartphones can have as much as 17.99\% average 
error compared to an external high-resolution power monitor 
and hence are not accurate enough to be used in SPMD. Thus we focus 
our study using the power monitor. 
}
%
\item 
However, when limiting SPMD within each of the app scenarios, creating 
one equation at a second time scale 
cannot create system of equations with enough diversity for 
the regression solver to output any meaningful power model parameters.
%
\cut{Despite adding many constraints to the regression solver,
the GPU Idle power is as high as the GPU Busy power. }
The key reason is that the CPU/GPU usage for the app scenarios such 
as games are highly repetitive; 
the systems of equations often have a single large singular value.  
\item 
We further explore if we zoom to smaller time scale, by setting each 
equation at the millisecond time scale
to explicitly exploit the known diversity of GPU usage, can improve 
the diversity of the system of equations. Our results show that doing 
so can 
make the regression solver to sometimes output meaningful model parameters, 
but the measurement noise at such fine time scale often overwhelm 
the equation diversity.
Further SPMD would not scale as a 1-minute app run would require solving thousands of systems of equations. 
\if 0
Our experiments show that the system still does not have enough diversity  
    at either the micro-scale, \ie with 2 equations in each rendering interval to exploit the GPU state diversity (Busy/Idle), or the nano-scale, \ie with 16 equations in each 16.7ms rendering interval, for two reasons: (1) the component usage lacks diversity,
    and (2) the noise in the coefficient measurements overshadows the diversity of utilization coefficients in the equations. 
\fi
\end{itemize}
    
These findings from our exploration strongly suggest that while SPMD may be useful for energy modeling which only requires a good least-square fitting of the two sides of the energy equations~\cite{sesame:2011}, it remains extremely challenging to derive component-wise power model parameters . 

