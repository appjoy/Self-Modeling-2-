\section{Related Work}
\label{sec:related}

The large amount of prior work on power modeling can be categorized along three dimensions: 
model type, using external power monitor or self-metering, and model derivation process.

\paragraph{Utilization-based vs. FSM-based power models} 
The large number of power models for different components of 
smartphones can be grouped into two categories
based on the way it captures two different types of phone component power draw behaviors. 
The first category of power models known as utilization-based models
(\eg \cite{shye2009into,zhang2010accurate,hotpowermulticore}) 
captures the synchronous power draw behavior of  components ; where
a hardware component enters a certain power state and draws power for the duration
when it is actively used in that state, \eg the CPU running at 2 GHz 
% and change of utilization is what triggers the power state change of that component. 
Consequently, these models use the utilization of a hardware component as an input to the power models.
% ``trigger'' in modeling power states and state transitions.  

% Such models thus do not capture power behavior of
% modern wireless components that do not lead to active utilization such
% as the promotion and tail power behavior of 3G and
% LTE~\cite{rrc:imc2010,4glte}, and thus can incur high modeling error.

The second category known as non-utilization-based power models captures the asynchronous power 
draw behavior of hardware components where an app’s usage of a phone component
triggers that component to enter some power state and continues even beyond the duration of the app usage.
Non-utilization-based models capture such power behavior using finite state machines (FSMs),
\eg \cite{arunabimc09,energybasedRA,aro,whatsup:mobicom12,pathak:eurosys12,ding:sigmetrics13}
for WiFi, 3G and~\cite{4glte} for 4G/LTE. 
To derive such models, the built-in state machine of such hardware components, \eg the RRC states and
transitions in the LTE radio, is reverse-engineered and represented in an FSM
annotated with app utilization, \eg either packet-level traces~\cite{rrc:imc2010,4glte} or
networking system calls~\cite{pathak:eurosys12}, as the triggers for the state transitions.

\paragraph{Using external power monitor vs. self-metering}
Deriving the power models of a mobile device 
requires ground-truth power draw measurement, whether it is for one component as in case of TPMD or for several components as a whole as in case of SPMD. This can be done using an external power monitor such as Monsoon~\cite{powermonitor} or
using energy information from the built-in battery interface. 

PowerBooter~\cite{powerbooter:2010} proposed to use battery State-of-Charge (SoD) readings to infer the average power draw has a very long generation time. 
%
Sesame~\cite{sesame:2011} uses drop in the battery capacity  during a sampling interval  to infer average power draw but it shows a high error at 0.1 Hz.
% 
V-edge~\cite{vedge:2013} exploits the correlation between change in current and voltage  
to extrapolate  power draw (relative to a baseline) using only voltage readings.  
DevScope~\cite{devscope:2012} studies the approach towards overcoming the coarse granularity of current sensor readings to derive power models for one single component at a time.  
%
Since Android 6, Android phones provided interfaces for power or energy readings which however come
with limited accuracy as discussed in \S\ref{subsec:modelling_sensor}.



\paragraph{TPMD vs. SPMD}
The third dimension of power modeling is whether to derive the model of one phone component state at a time by isolating the measured power for each of the component state as in case of TPMD, or 
to derive the model for all the component states by solving a system of equations as in case of SPMD, as already discussed in \S\ref{subsec:back}.